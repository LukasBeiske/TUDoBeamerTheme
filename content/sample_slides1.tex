\begin{frame}{Hinweise}
  Zu diesem Theme
  \begin{itemize}
    \item Zur Installation des Themes müssen mindestens die Datei \texttt{beamerthemetudo.sty} und der Ordner \texttt{logos} in einen Ordner verschoben werden, in dem \LaTeX nach Paketen sucht.
      Dies können sein
      \begin{itemize}
        \item \texttt{TEXMFHOME/tex/latex/tudobeamertheme}. Den Wert von \texttt{TEXMFHOME} bekommen Sie über \texttt{kpsewhich --var-value TEXMFHOME}, üblicherweise ist dies \texttt{\$HOME/texmf}.
        \item Der gleiche Ordner in dem Sie Ihr Dokument kompilieren.
        \item Ein beliebiger Ordner, der in der Variablen \texttt{TEXINPUTS} enthalten ist.
      \end{itemize}
  \end{itemize}
  
  Oneliner zur Installation:\\
  \texttt{\footnotesize\$ cd `kpsewhich --var-value TEXMFHOME` \&\& git clone https://github.com/maxnoe/tudobeamertheme}

  \medskip
  Allgemein zu Beamer und Latex:
  \begin{itemize}
    \item Umfangreicher \LaTeX-Kurs von PeP et Al. \\
      \url{http://toolbox.pep-dortmund.org/notes}
    \item Latex-Beamer Dokumetation:\\
    \url{http://www.ctan.org/pkg/beamer}
  \end{itemize}
\end{frame}